%% -----------------------------------------------------------------
%% This file uses UTF-8 encoding
%%
%% For compilation use following command:
%% latexmk -pdf -pvc -bibtex thesis
%%
%% -----------------------------------------------------------------
%%                                     _     _      
%%      _ __  _ __ ___  __ _ _ __ ___ | |__ | | ___ 
%%     | '_ \| '__/ _ \/ _` | '_ ` _ \| '_ \| |/ _ \
%%     | |_) | | |  __/ (_| | | | | | | |_) | |  __/
%%     | .__/|_|  \___|\__,_|_| |_| |_|_.__/|_|\___|
%%     |_|                                          
%%
%% -----------------------------------------------------------------

\documentclass{kithesis}

% Additional packages
\usepackage[main=english,slovak]{babel}
% For thesis written in English just change the order of languages:
% \usepackage[main=english,slovak]{babel}

\usepackage{listings}  % for source code
% Listings settings
% See for details: https://en.wikibooks.org/wiki/LaTeX/Source_Code_Listings
\lstset{
    basicstyle=\small\ttfamily,  % smaller typewriter font
    showstringspaces=false       % don't show spaces in string
}

% Location of file with bibliography resources
\addbibresource{chapters/bibliography.bib}

% Variables
%\thesisspec{figures/thesisspec.png} 

\title{Advanced prediction models for sales forecasting}{Advanced prediction models for sales forecasting}

\author[Bc.]{Aleš}{Jandera}
\supervisor{doc. Ing. Tomáš Škovránek, PhD.} %veduci prace
%\consultant{Donald E. Knuth} %konzultant
%\college{University of Žilina}{Žilinská univerzita} %univerzita
%\faculty{Faculty of Electrical Engineering and informatics}{Fakulta elektrotechniky a informatiky} %fakulta
%\department{Department of Computers and Informatics}{Katedra počítačov a informatiky} %katedra
%\departmentacr{DCI}{KPI} % skratka katedry
%\thesis{Master thesis}{Diplomová práca} %typ prace
\submissiondate{30}{4}{2023}
%\fieldofstudy{9.2.1 Informatika}
%\studyprogramme{Informatika}
%\city{Košice} %mesto
\keywords{Mathematic modeling, forecasting, linear prediction}{Matematicke modelovanie, predpoved, linearna predikcia}
%\declaration{som nepodvadzal}

\abstract{%
    % english 
	Sales forecasting can be divided into two main categories: short-term and long-term forecasting.
    Short-term forecasting is generally done on a weekly or monthly basis. Long-term forecasting is done on a quarterly or annual basis.
    There are many different methods that can be used for sales forecasting.
    The most common method is trend analysis. Trend analysis looks at past sales data to identify
    patterns and trends that can be used to predict future sales.
    Other methods include regression analysis and time series analysis.
    Advance prediction modeling is a type of long-term forecasting.
    It uses historical data and statistical techniques to predict future sales.
    Advance prediction modeling is often used by companies to make strategic decisions about inventory, pricing, and marketing.
}{%
    % slovak 
	Prognózy predaja možno rozdeliť do dvoch hlavných kategórií: krátkodobá a dlhodobá prognóza.
    Krátkodobé prognózy sa vo všeobecnosti vykonávajú týždenne alebo mesačne. Dlhodobé prognózy sa robia štvrťročne alebo ročne.
    Existuje mnoho rôznych metód, ktoré možno použiť na predpovedanie predaja.
    Najbežnejšou metódou je analýza trendov. Analýza trendov sa zameriava na údaje o minulých predajoch, aby identifikovala vzory a trendy,
    ktoré možno použiť na predpovedanie budúceho predaja. Ďalšie metódy zahŕňajú regresnú analýzu a analýzu časových radov.
    Predbežné predikčné modelovanie je typ dlhodobého predpovedania.
    Na predpovedanie budúceho predaja využíva historické údaje a štatistické techniky.
    Pokročilé predikčné modelovanie často používajú spoločnosti na strategické rozhodnutia o zásobách, cenách a marketingu.
}

\acknowledgment{Na tomto mieste by som rád poďakoval svojmu vedúcemu práce za jeho čas a odborné vedenie počas riešenia mojej záverečnej práce.

Rovnako by som sa rád poďakoval svojim rodičom a priateľom za ich podporu a povzbudzovanie počas celého môjho štúdia.
    
V neposlednom rade by som sa rád poďakoval pánom \textit{Donaldovi E. Knuthovi} a \textit{Leslie Lamportovi} za typografický systém \LaTeX, s ktorým som strávil množstvo nezabudnuteľných večerov.}

% if you want to work only on selected chapters
%\includeonly{chapters/analyza} %,chapters/synteza}

% Load acronyms
\input{acronyms}


%% -----------------------------------------------------------------
%%          _                                       _   
%%       __| | ___   ___ _   _ _ __ ___   ___ _ __ | |_ 
%%      / _` |/ _ \ / __| | | | '_ ` _ \ / _ \ '_ \| __|
%%     | (_| | (_) | (__| |_| | | | | | |  __/ | | | |_ 
%%      \__,_|\___/ \___|\__,_|_| |_| |_|\___|_| |_|\__|
%%                                                      
%% -----------------------------------------------------------------

\begin{document}
%% Title page, abstract, declaration etc.:
\frontmatter{}

%% List of code listings, if you are using package minted
%\listoflistings

%\pagenumbering{arabic}

%% Chapters
% !TEX root = ../thesis.tex

\chaptermark{Introduction}
\phantomsection
\addcontentsline{toc}{chapter}{Introduction}

\chapter*{Introduction}

Linear prediction is a method used in signal processing to predict future values of a time series based on past observations.
The technique is based on the assumption that the signal can be modeled as a linear combination of past values and a noise term.
Long-term linear prediction refers to the application of this method to predict values over a longer period of time, such as months or years.
It requires a greater amount of data and is more complex than short-term prediction, but can be useful in areas such as stock market forecasting and
weather prediction. The goal of this master thesis is to developing new algorithms and mathematical models to improve the accuracy of long-term predictions
in sales forecasting. Matlab \footnote{MATLAB is a fourth-generation programminglanguage and numerical analysis environment.
Uses for MATLAB include matrix calculations, developing and running algorithms, creating user interfaces (UI) and data visualization.}
livescript~\cite{livescript} will be used as development enviroment.

\section*{Task formulation}

Proposed a mathematical model and alghoritms for sales forcasting based on long-term prediction with improved Levinson - Durbin scheme 
which should have better performance and acuraccy than known linear prediction mechanism.

% !TEX root = ../thesis.tex

\chapter{Analytická časť}

Analytická časť záverečnej práce analyzuje existujúce podobné prístupy k~riešeniu stanoveného problému. Autor práce musí uviesť v~tejto časti existujúce prístupy a riešenia, pričom musí zaujať stanovisko k~týmto prístupom a riešeniam a opísať ich výhody a nedostatky. Prevažne v~tejto časti autor používa odkazy na použité zdroje. Autor v~analýze nepreberá odseky z~cudzích prác ale uvádza prevažne vlastné postoje podložené odkazmi na literatúru. Analytická časť práce by teda nemala byť len povrchným prepisom základných informácií z~Wikipédie alebo zo stránok opisovaných nástrojov. Je potrebné aby bola analýza podporená aj experimentmi ak to umožňuje téma práce (napr. vyskúšam softvér). Vďaka popisu existujúcich riešení autor pochopí problematiku, viac sa nad riešeniami zamyslí, usporiada si ich, zistí ich kladné a záporné vlastnosti, z~čoho potom postupne vyplynie návrh vlastného riešenia v~syntetickej časti. Analytická časť tvorí zvyčajne ¼ jadra práce.

Analytickú časť je možné rozdeliť na niekoľko kapitol, ktoré budú venované rôznym analyzovaným témam. Názvy kapitol majú zodpovedať tomu, čo je v~kapitole opisované. Napríklad ak v~práci analyzujete súčasný stav v~oblasti medzigalaktických letov, namiesto všeobecného názvu "`Analýza súčasného stavu"' by mal byť použiťý názov analyzovanej témy --- "`Medzigalaktické lety"'.


\blindtext

\blindtext

% !TEX root = ../thesis.tex

\chapter{Syntactic part} \label{sec:methodology}
Based on the Analytical part~\ref{sec:analytical} let us create new mathematical models and approaches to made a fast and accuracy sales forcasting
consist of long-therm linear prediction with individual weights calculated for each period all based on Levinson-Durbin scheme caled Extended linear prediction (ELP)
We expect to get better results than by using prediction based on short-term or long-term standard linear prediction (see section~\ref{sec:lp}).
Finally, our approach will return future values for sales companies based on previous data with better aberration than linear prediction has.
\section{Extended long-term prediction} \label{sec:models}
\section{Weights for each period} \label{sec:models}
\section{AI principles to detect best order of linear prediction} \label{sec:models}
\section{Combining all principles to forecast process} \label{subsec:combining_models}
% !TEX root = ../thesis.tex

\chapter{Evaluation} \label{evaluation}
\section{Experiment} \label{sec:experiment}
Lets prepare the experiment to validating a long-term linear prediction model for online store orders. We will use experimentation by these steps:\\
\begin{enumerate}
    \item Define the problem and objectives: Clearly define the problem and objectives that the long-term linear prediction model is intended to solve.
    In this case, the objective is to predict online store orders over a long period of time using a linear model.
    \item Gather data: Collect historical data on online store orders, such as order dates, order amounts, and other relevant variables.
    This data will be used to train and validate the long-term linear prediction model. Our data was collect from online system storePredictor
    which is available on storepredictor.com and data are anonymized and pseudonimized to be used for next calculations.
    \item Prepare the data: Clean and preprocess the data to ensure that it is consistent and suitable for use in the long-term linear prediction model.
    This may involve tasks such as removing duplicates, handling missing values, and transforming variables as needed.
    Preprocessing of our data is described in \ref{subsec:preprocessing}
    \item Develop the long-term linear prediction model: Using the historical data, develop a long-term linear prediction model that can
    forecast online store orders over a specified time period, which is detailed described in \ref{sec:extlonglp} and practically applied in \label{subsec:calculate_models}
    \item Validate the model: Once the model is developed, it needs to be validated to ensure that it is accurate and effective.
    This can be done by comparing the model's predictions with actual online store orders over a specified time period, described in \label{subsec:validatibg_models}
    \item Evaluate the results: Analyze the results of the experiment to determine the accuracy of the long-term linear prediction model.
    This will involve calculating metrics such as the mean square error (MSE), r-squared ($R^2$) and the root mean square error (RMSE) \ref{subsec:experimentResults}.
    \item During solving previous steps the model was redefined and revalidate when the results are not satisfactory, refine the model and repeat the validation
    process until an accurate and effective long-term linear prediction model is developed.
\end{enumerate}
In summary, to prepare an experiment to validate a long-term linear prediction model for online store orders, you need to
define the problem and objectives, gather and prepare the data, develop the model, validate it, evaluate the results, and refine and revalidate
the model if necessary.
    \subsection{Preprocessing of input data} \label{subsec:preprocessing}
    Preprocessing the data is an important step in preparing the data for a linear order prediction model. Here are some common
    steps to preprocess the data for linear order prediction:

    \begin{enumerate}
        \item Data cleaning: Remove any irrelevant data or duplicate records in the dataset. In our purpose is to clean the unsuccessfull orders,
        returned orders and fraud orders from competitors to detect power of the store.
        \item Handling missing data: for fill in any missing values in the dataset machine learning algorithms K-Nearest Neighbors (KNN) \ref{sec:knn} will be used
        to check the dataset.
        \item Feature selection: Select the relevant features (predictors) that are likely to have a strong influence on the order
        prediction. This may include variables such as customer demographics, purchase history, product attributes, and marketing campaigns.
        \item Feature scaling: Scale the features so that they are on the same scale to ensure that each feature has equal importance.
        This can be done using normalization or standardization techniques.
        \item Handling categorical variables: Convert categorical variables into numerical values using techniques
        such as one-hot encoding, ordinal encoding, or label encoding.
        \item Dimensionality reduction: If the dataset contains many features, use dimensionality reduction
        techniques such as principal component analysis (PCA) or linear discriminant analysis (LDA) to reduce the number of features and simplify the model.
        \item Handling outliers: Detect and handle any outliers in the dataset using appropriate techniques
        such as Z-score, Tukey’s method, or machine learning algorithms such as Isolation Forest or Local Outlier Factor.
    \end{enumerate}
    Overall, the goal of this steps for linear order prediction experiment is to ensure that the data is clean, complete, and properly
    formatted for use in the linear prediction model. This helps to improve the accuracy and effectiveness of the model in predicting online store orders.
    \subsection{Model for sales forecasting} \label{subsec:calculate_models}
    Let's prepare 4 models to test our theory and result expectation.  Set up MATLAB live script to prepare our dataset, calculate statistical variables and
    develop several versions of linear predictors to get reults of our experiment. We will use this models:
    \begin{enumerate}
        \item Short-term linear prediction 
        Based on linear prediction \ref{subsec:shortlp} we calculate optimal coeficients by \ref{subsec:levinson} and made prediction over our dataset.
        This model is described in equation \ref{eq:slp}.

        \item Extended short-term linear prediction 
        This model is based on linear prediction \ref{subsec:shortlp} with optimal parameters by levinson schema \ref{subsec:levinson}.
        To compare our developed model \label{subsec:extlonglp} and we will prepare similar principle as in equation \ref{eq:eltlp} to the short-term linear
        prediction with results:
        
        \begin{equation} \label{eq:slp}
            \begin{aligned}
                &\hat{y} = \sum_{j=1}^{n} a(j)*y(n-j) * \gamma(t)
            \end{aligned}
        \end{equation}
        \begin{itemize}
            \item $\hat{y}$ is the value of the predicted order
            \item $a(1), a(2), ..., a(n)$ are the prediction coefficients
            \item $y(n-1), y(n-2), ..., y[n-j]$ are the past values of the signal that are used to make the prediction
        \end{itemize}

        \item Long-term linear prediction 
        To compare more linear prediction and check theorem about application of long-term prediction on sales data we will prepare the model
        based on \label{subsec:longlp} and use the equation \ref{eq:ltlp}.

        \item Extended long-term linear prediction
        The last model in our comparison will be our developed model which is based on long-term linear prediction describe in \ref{subsec:longlp} and
        use weights coefficients \ref{sec:weights} to get better results in economics data. This model is describe in equation \ref{eq:eltlp}.
    \end{enumerate}
    % reference to equations: eq:eltlp, 
    All the models will be used dataset prepared in \ref{subsec:preprocessing} and results of prediction is described in \ref{subsec:experimentResults}.
    \subsection{Results} \label{subsec:experimentResults}
        \subsubsection{Short-term linear prediction} \label{subsec:res_slp}
        \begin{figure}[h]
            \centering
            \begin{minipage}{0.45\textwidth}
                \centering
                \includegraphics[width=1\textwidth]{figures/expLP.png}
                \caption{Results of short-term linear prediction.}
                \label{fig:slpres}
            \end{minipage}\hfill
            \begin{minipage}{0.45\textwidth}
                \centering
                \includegraphics[width=\textwidth]{figures/expMseLP.png}
                \caption{MSE of short-term linear prediction.}
                \label{fig:slpmse}
            \end{minipage}
        \end{figure}
        \newpage

        \subsubsection{Extended short-term linear prediction} \label{subsec:res_estlp}
        \begin{figure}[h]
            \centering
            \begin{minipage}{0.45\textwidth}
                \centering
                \includegraphics[width=1\textwidth]{figures/expELP.png}
                \caption{Results of extended short-term linear prediction.}
                \label{fig:eslpres}
            \end{minipage}\hfill
            \begin{minipage}{0.45\textwidth}
                \centering
                \includegraphics[width=1\textwidth]{figures/expMseELP.png}
                \caption{MSE of extended short-term linear prediction.}
                \label{fig:eslpmse}
            \end{minipage}
        \end{figure}
        \newpage

        \subsubsection{Long-term linear prediction} \label{subsec:res_ltlp}
        \begin{figure}[h]
            \centering
            \begin{minipage}{0.45\textwidth}
                \centering
                \includegraphics[width=1\textwidth]{figures/expLTLP.png}
                \caption{Results of long-term linear prediction.}
                \label{fig:ltlp}
            \end{minipage}\hfill
            \begin{minipage}{0.45\textwidth}
                \centering
                \includegraphics[width=1\textwidth]{figures/expMseLTLP.png}
                \caption{MSE of long-term linear prediction.}
                \label{fig:ltlpmse}
            \end{minipage}
        \end{figure}
        \newpage
        \subsubsection{Extended long-term linear prediction} \label{subsec:res_eltlp}
        \begin{figure}[h]
            \centering
            \begin{minipage}{0.45\textwidth}
                \centering
                \includegraphics[width=1\textwidth]{figures/expELTLP.png}
                \caption{Results of extended long-term linear prediction.}
                \label{fig:eltlpres}
            \end{minipage}\hfill
            \begin{minipage}{0.45\textwidth}
                \centering
                \includegraphics[width=1\textwidth]{figures/expMseELTLP.png}
                \caption{MSE of long-term linear prediction.}
                \label{fig:eltlpmse}
            \end{minipage}
        \end{figure}
        \newpage
        \subsubsection{Comparison of the predictors} \label{subsec:res_comparison}
        \begin{table}[!ht]
            \centering
            \begin{tabular}{|l|c|c|c|}
                \hline
                Model & $R^2$ & RMSE & MSE \\
                \hline
                Short-term linear prediction & 0.8872 & 40.6215 & 1650.1 \\
                Short-term extended linear prediction & 0.8881 & 40.4662 & 1637.5 \\
                Long-term linear prediction & 0.9160 & 35.0163 & 1223.3 \\
                Long-term extended linear prediction & 0.9171 & 34.7782 & 1206.7 \\
                \hline
            \end{tabular}
            \caption{Comparison of linear prediction models}
            \label{tab:model_comparison}
        \end{table}

        \begin{figure}[!ht]
            \begin{minipage}{0.45\textwidth}
                \centering
                \includegraphics[width=1\textwidth]{figures/expCompLP.png}
                \caption{MSE success of prediction for short-term linear prediction.}
                \label{fig:eltlpres}
            \end{minipage}\hfill
            \begin{minipage}{0.45\textwidth}
                \centering
                \includegraphics[width=1\textwidth]{figures/expCompELP.png}
                \caption{MSE success of prediction for extended short-term linear prediction.}
                \label{fig:eltlpmse}
            \end{minipage}
        \end{figure}

        \begin{figure}[!ht]
            \begin{minipage}{0.45\textwidth}
                \centering
                \includegraphics[width=1\textwidth]{figures/expCompLTLP.png}
                \caption{MSE success of prediction for long-term linear prediction.}
                \label{fig:eltlpres}
            \end{minipage}\hfill
            \begin{minipage}{0.45\textwidth}
                \centering
                \includegraphics[width=1\textwidth]{figures/expCompELTLP.png}
                \caption{MSE success of prediction for long-term linear prediction.}
                \label{fig:eltlpmse}
            \end{minipage}
        \end{figure}

% !TEX root = ../thesis.tex

\chapter{Summary} \label{summary}
The aim of this master's thesis was to improve the accuracy of linear prediction models for sales
and financial forecasting using modern machine learning techniques. To achieve this goal, a mathematical
model based on long-term linear prediction was created, and periodical weights were implemented
to improve the accuracy of the linear prediction models.\\
\\
Two main techniques were employed to apply machine learning to linear prediction. The first technique,
called feature engineering, involved using machine learning to extract meaningful features from
the input signal, which were then used as inputs for a linear prediction model. Several feature
engineering techniques were used, including principal component analysis, wavelet transform, and
Fourier transform.\\
\\
The second technique involved model selection and training, which used machine learning to choose
the best linear model for the prediction task and estimate its parameters from the data. Common machine
learning algorithms such as linear regression, support vector regression, and artificial neural networks
were used for model selection and training.\\
\\
The ultimate goal of long-term linear prediction was to estimate future values of a signal or time
series based on its past values using a linear model. The extended long-term linear prediction method
proved to be the most effective, with comparison parameters such as R2 getting 0.9171, RMSE of 34.7782, and MSE ending
with a value of 1206.7.\\
\\
The dataset used for the analysis was collected in 2022 as real orders. The first part of the dataset
was used to train the models, and the second part was used to validate the results.\\
\\
The results of this study were achieved in collaboration with the E-commerce Association o.z,
the creator of the storePredictor platform. The organization provided valuable business insights into
the industry, which helped set mechanisms for periodical weights to extend linear prediction models
with minimal deflection. As a result, the next steps should focus on clearly defining and validating
mechanisms to set independent industry weights only on ordinary accessible data without consultation
with business owners.\\
\\
In conclusion, this study successfully improved the accuracy of linear prediction models for sales
and financial forecasting by utilizing modern machine learning techniques. The results of this study
have practical implications for businesses, as accurate forecasting can help them make better decisions,
increase profits, and achieve their goals.

\chapter{Resume} \label{resume}
    \section{Analyza} \label{sk:analytic}
    Lineárna predikcia je štatistická metóda používaná na predpovedanie budúcich hodnôt na základe historických dát pre
    identifikaciu parametrov pouziva Durbin-Levinsonov algoritmus coz je metóda riešenia lineárnej predikcie pre
    autoregresívne (AR) modely1, ktoré sú modely, kde aktuálny výstup závisí od predchádzajúcich výstupov. Algoritmus
    rieši problém lineárnej predikcie nájdením koeficientov AR modelu, ktoré minimalizujú chybu predikcie. Výsledné AR
    koeficienty môžu byť použité na predpovedanie budúcich hodnôt na základe minulých pozorovaní. Táto metóda by sa mala
    používať s použitím vzoru lineárneho vzťahu medzi nezávislými a závislými premennými. Tu je základný prehľad krokov
    pri použití lineárnej predikcie na predpovedanie dát o predaji:\\
    \\
    \begin{enumerate}
        \item Zbierajte dáta o predaji: Získajte historické dáta o predaji produktu alebo služby, ktorú chcete predpovedať.
        \item Vykreslite dáta: Vykreslite dáta o predaji v čase, aby ste vizuálne preskúmali trend a identifikovali
        akékoľvek vzory.
        \item Vyberte model: Vyberte vhodný lineárny model na zobrazenie vzťahu medzi nezávislými a závislými
        premennými v dátach. Napríklad by ste si mohli vybrať jednoduchý lineárny regresný model.
        \item Natrénujte model: Natrénujte vybraný model na historických dátach o predaji pomocou metódy ako najmenšie štvorce.
    \end{enumerate}
    Autoregresívne (AR) modely sú modely časových radov, ktoré popisujú vzťah medzi súčasnou hodnotou premennej a jej
    minulými hodnotami. V autoregresívnom modeli je každá pozorovanie modelované ako lineárna kombinácia minulých
    pozorovaní, s váhami nazývanými AR koeficienty. AR modely sú široko používané v rôznych oblastiach, ako je ekonomika,
    inžinierstvo a financie, na modelovanie a predpovedanie časových radových dát. Poradie AR modelu, označované
    ako "p", sa vzťahuje na počet minulých hodnôt použitých na predpovedanie súčasnej hodnoty. Napríklad AR(1) model
    používa len predchádzajúce pozorovanie na predpovedanie súčasnej hodnoty, zatiaľ čo AR(2) model používa predchádzajúce
    dve pozorovania.\\
    \\
    Vykonajte predpovede: Použite trénovaný model na predpovedanie budúcich predajových dát. Môžete chcieť generovať
    predpovede na niekoľko mesiacov alebo rokov dopredu.
    \begin{enumerate}
        \item Zhodnoťte model: Posúďte presnosť predpovedí porovnaním s faktickými predajovými dátami. Použite metriky,
        ako je priemerná absolútna chyba alebo koreňová stredná štvorcová chyba, na kvantifikáciu výkonu modelu.
        \item Vylepšite model: Ak je to potrebné, vylepšite model pridaním ďalších nezávislých premenných alebo
        transformáciou existujúcich premenných.
        \item Opakujte kroky trénovania a hodnotenia, kým nebudete mať model, ktorý poskytuje presné prognózy.
        \item Pre výpočet posunu v dlhodobom predpovedaní sa môže použiť autokorelačná metóda. V tejto práci sa
        vyvinie neurónová sieť na identifikáciu posunu a podobný mechanizmus pre optimálne určenie poradia.
    \end{enumerate}
    \textbf{Modely používané pre predajové dáta} \\
    Existuje niekoľko matematických modelov používaných pre predpovedanie predaja, vrátane:\\
    \begin{enumerate}
        \item Modely časových radov: Tieto modely sa používajú na analýzu a predpovedanie predajových dát v čase,
        ako sú sezónne vzorce, trendy a fluktuácie. Príklady zahŕňajú ARIMA (AutoRegressive Integrated Moving Average),
        SARIMA (Seasonal ARIMA) a exponenciálne vyhladzovanie.
        \item Regresné modely: Tieto modely používajú historické údaje na určenie vzťahu medzi predajom a jednou alebo
        viacerými nezávislými premennými, ako sú cena, propagácia a reklama. Príklady zahŕňajú lineárnu regresiu,
        logistickú regresiu a viacnásobnú regresiu.
        \item Modely stromových rozhodnutí: Tieto modely používajú štruktúru stromu na rozhodovanie založené na vzťahu medzi
        predajom a viacerými nezávislými premennými. Príklady zahŕňajú CART (Klasifikácia a regresia stromu) a náhodný les.
        \item Modely strojového učenia: Tieto modely používajú algoritmy ako neurónové siete a stroje s podpornými
        vektormi na predikovanie na základe vzorov v údajoch.
    \end{enumerate}
    \textbf{Neurónove sieťe} \\
    Neurónová sieť je druh algoritmu strojového učenia inšpirovaný štruktúrou a funkciou biologických neurónov v
    ľudskom mozgu. Skladá sa z prepojených uzlov, nazývaných neuróny, ktoré sú usporiadané do vrstiev. Vstupná vrstva
    prijíma surové dáta, ako sú obrázky alebo text, a prenáša ich do skrytých vrstiev, ktoré vykonávajú výpočty a váhy
    sa aplikujú na vstupné dáta pre vytvorenie predikcie. Nakoniec výstupná vrstva produkuje konečnú predikciu
    alebo klasifikáciu.\\
    \\
    Ako môžete vidieť na obrázku\ref{fig:perceptron}, každý vstup $Xn$ by mal byť správne ohodnotený určitou
    váhou $W$ n predtým, než všetky signály vstúpia do sumovacej fázy. Potom sa vážené súčty prenášajú do aktivačnej
    jednotky produkujúcej výstupný signál neurónu.\\
    \\
    Neurónové siete sa trénujú na veľkých dátových súboroch pomocou procesu nazývaného spätné šírenie chyby, ktorý
    upravuje váhy a sklon neurónov, aby minimalizoval rozdiel medzi predpovedaným výstupom a skutočným výstupom.
    Akonáhle je neurónová sieť natrénovaná, môže sa použiť na predpovedanie nových dát.\\
    \\
    Neurón je základnou stavebnou jednotkou neurónovej siete, známy aj ako umelej neurón alebo perceptrón.
    Modeluje sa podľa biologického neurónu v ľudskom mozgu, ktorý prijíma vstupné signály z iných neurónov,
    spracováva ich a posiela výstupné signály do ďalších neurónov.\\
    \\
    V neurónovej sieti neurón prijíma vstup od iných neurónov alebo priamo od vstupných dát, aplikuje na vstup
    matematickú funkciu a produkuje výstup, ktorý sa posiela do ďalších neurónov v sieti. Vstupom do neurónu je
    zvyčajne vektor čísel a každý vstup sa násobí príslušnou váhou.\\
    \\
    Potom neuron sčíta vážené vstupy, pridáva sklon a aplikuje aktivačnú funkciu na výsledok. Úlohou aktivačnej
    funkcie je zaviesť nelinearitu do neurónu, čo umožňuje neuronovej sieti naučiť sa zložité vzorce a vzťahy v dátach.
    Existuje niekoľko rôznych typov aktivačných funkcií, ktoré sa môžu použiť, ako napríklad sigmoidná funkcia,
    ReLU (Rectified Linear Unit) funkcia a tanh (hyperbolická tangens) funkcia.\\
    \\
    Výstup neurónu sa zvyčajne posúva do ďalších neurónov v nasledujúcej vrstve neurónovej siete. Váhy a sklon
    neurónov sa počas trénovania prispôsobujú technikou spätného šírenia chyby, ktorá zahŕňa výpočet gradientu chyby
    vzhľadom na váhy a aktualizovanie ich pomocou optimalizačného algoritmu, ako je stochastický gradientový zostup.\\
    \\
    Celkovo neuróny v neurónovej sieti spolupracujú na učení vzorcov a vzťahov vstupných dát a produkujú výstup,
    ktorý sa môže použiť pre rôzne úlohy, ako je klasifikácia, regresia a predikcia.\\
    \\
    Neurónové siete sa úspešne uplatňujú v širokej škále oblastí, vrátane rozpoznávania obrazov a reči, spracovania
    prirodzeného jazyka a autonómnych vozidiel, medzi inými.
    \section{Synteza}
    Vychadzajme z analytickej časti \ref{sk:analytic}, vytvorme nové matematické modely a prístupy, aby sme mohli
    vykonať rýchle a presné predpovede predaja,ktoré sa skladajú z long-term lineárnej predikcie s individuálnymi
    váhami vypočítanými pre každé obdobie, založené na Levinson-Durbinovej schéme nazývanej Extended Linear Prediction
    (ELP). Očakávame lepšie výsledky než pri použití predikcie založenej na krátkodobej alebo long-term štandardnej
    lineárnej predikcii (viď časť 1.4). Nakoniec, náš prístup bude vracať budúce hodnoty pre predaj spoločností na základe
    predchádzajúcich dát s lepšou odchýlkou, než to dokáže lineárna predikcia.\\
    \\
    Pre vytvorenie matematického modelu na predikciu predajných dát s periodickými trendmi môžete použiť model
    sezónnej ARIMA (SARIMA). Tento model zohľadňuje sezónne variácie v dátach a používa autoregresívne a kĺzavé
    priemerové členy na zachytenie vzorov a trendov v dátach.\\
    \\
    Pre svoj účel som vytvoril rozšírenú dlhodobú predikciu, ktorá sa vysporiada so sezónnymi a opakujúcimi sa vzormi
    v ekonomických dátach. Tento korekčný mechanizmus sa používa na zohľadnenie historických vrcholov v grafoch cez dataset.
    Táto jednoduchá korekcia zvyšuje presnosť modelu a získava lepšiu odpoveď vďaka psychologickým,
    sociologickým a marketingovým aspektom v datasete.\\
    \\
    Na nastavenie periodických váh je potrebné vytvoriť vektor korekčných parametrov z pôvodného datasetu pomocou
    štatistických parametrov mediánu a štandardnej odchýlky z datasetu. Ako základnú rovnicu použijeme rovnicu pre
    dlhodobú predikciu 4.1 s novými váhami a získame tak lepšie výsledky.
    \section{Experiment}
    Pripravme experiment na overenie dlhodobého lineárneho predikčného modelu objednávky internetového obchodu.
    Použijeme experimentovanie podľa týchto krokov:
    \begin{enumerate}
        \item Definujte problém a ciele: Jasne definujte problém a predmet ktoré má vyriešiť dlhodobý lineárny
        predikčný model. V tomto Cieľom je predpovedať objednávky v internetovom obchode na dlhé časové obdobie
        pomocou lineárneho modelu.
        \item  Zhromažďovanie údajov: Zhromažďujte historické údaje o objednávkach online obchodov, ako je napríklad
        objednávka dátumy, sumy objednávok a ďalšie relevantné premenné. Tieto údaje budú použité trénovať a overovať model
        long-term lineárnej predikcie. Naše údaje boli zbierať z online systému storePredictor, ktorý je dostupný na
        storepredictor.com a údaje sú anonymizované a pseudonimizované, aby sa mohli použiť na ďalšie účely výpočty.
        \item  Pripravte údaje: Vyčistite a predspracujte údaje, aby ste sa uistili, že sú konzistentné stan a vhodné na
        použitie v dlhodobom lineárnom predikčnom modeli. Toto môže zahŕňajú úlohy, ako je odstraňovanie duplikátov,
        spracovanie chýbajúcich hodnôt a transformovať premenné podľa potreby. Predspracovanie našich údajov
        je popísané v \ref{subsec:preprocessing}.
        \item  Vytvorte model long-term lineárnej predikcie: Pomocou historických údajov, vyvinúť dlhodobý lineárny
        predikčný model, ktorý dokáže predpovedať internetový obchod objednávky počas určitého časového obdobia, ktoré
        je podrobne popísané v bode 4.4 a prakticky aplikovaný v TODO
        \item  Overenie modelu: Keď je model vytvorený, je potrebné ho overiť aby sa zabezpečilo, že je presný a účinný.
        \item Vyhodnoťte výsledky: Analyzujte výsledky experimentu, aby ste určili presnosť modelu long-term lineárnej
        predikcie. To bude zahŕňať výpočet metrík, ako je stredná štvorcová chyba (MSE), r-kvadrát (R2) a stredná
        kvadratická chyba (RMSE) 5.1.3. TODO
        \item Počas riešenia predchádzajúcich krokov bol model predefinovaný a revalidovaný, keď výsledky nie sú uspokojivé,
        upravte model a zopakujte validáciu až kým nevznikne presný a efektívny dlhodobý lineárny predikčný model vyvinuté.
    \end{enumerate}
    Stručne povedané, pripraviť experiment na overenie long-term lineárnej predpovede model pre objednávky v
    internetovom obchode, musíte definovať problém a ciele, zhromažďovať a pripravovať údaje, vyvíjať model,
    overovať ho, hodnotiť výsledky, a ak je to potrebné, model spresnite a znovu overte.
    \section{Zaver}
    Cieľom tejto diplomovej práce bolo zlepšiť presnosť lineárnej predikcie
    modely pre predaj a finančné prognózy využívajúce moderné technológie strojového učenia
    niky. Na dosiahnutie tohto cieľa sme vytvorili matematický model založený na dlhodobom lineárnom
    bola vytvorená predpoveď a boli implementované periodické váhy na zlepšenie
    presnosť lineárnych predikčných modelov.\\
    \\
    Na aplikáciu strojového učenia na lineárne predikcia. Prvá technika, nazývaná inžinierstvo funkcií,
    zahŕňala použitie stroja naučiť sa extrahovať zmysluplné vlastnosti zo vstupného signálu, ktoré boli vtedy
    používané ako vstupy pre lineárny predikčný model. Niekoľko technických funkcií -
    boli použité niques, vrátane analýzy hlavných komponentov, vlnkovej transformácie,
    a Fourierova transformácia.\\
    \\
    Druhá technika zahŕňala výber modelu a tréning, ktorý využíval stroj
    naučiť sa vybrať najlepší lineárny model pre predikčnú úlohu a odhadnúť jej
    parametre z údajov. Bežné algoritmy strojového učenia, ako napríklad lineárne
    bola použitá regresia, podporná vektorová regresia a umelé neurónové siete
    pre výber modelu a tréning.\\
    \\
    Konečným cieľom dlhodobej lineárnej predikcie bolo odhadnúť budúce hodnoty
    signálu alebo časového radu na základe jeho minulých hodnôt pomocou lineárneho modelu. bývalý
    tendencia dlhodobej lineárnej predikčnej metódy sa ukázala ako najefektívnejšia, s
    porovnávacie parametre, ako napríklad R2 získavanie 0,9171, RMSE 34,7782 a MSE
    končiace hodnotou 1206,7.\\
    \\
    Súbor údajov použitý na analýzu sme zozbierali v roku 2022 ako skutočné objednávky. Prvu
    časť súboru údajov bola použitá na trénovanie modelov a druhá časť bola použitá na
    overiť výsledky.\\
    \\
    Výsledky tejto štúdie boli dosiahnuté v spolupráci s E-commerce
    Združenie o.z, tvorca platformy storePredictor. Organizácia pre-
    poskytli cenné obchodné poznatky v tomto odvetví, ktoré pomohli nastaviť mechanizmy
    pre periodické váhy na rozšírenie modelov lineárnej predikcie s minimálnym priehybom
    cie. V dôsledku toho by sa ďalšie kroky mali zamerať na jasné definovanie a overenie
    mechanizmov na nastavenie nezávislých odvetvových váh len na bežne dostupné údaje
    bez konzultácie s majiteľmi firiem.\\
    \\
    Na záver, táto štúdia úspešne zlepšila presnosť lineárnej predikcie
    modely pre predaj a finančné prognózy s využitím moderného strojového učenia
    techniky. Výsledky tejto štúdie majú praktické dôsledky pre podniky,
    keďže presné predpovede im môžu pomôcť robiť lepšie rozhodnutia, zvýšiť zisky a
    dosiahnuť svoje ciele.


% good linebraking of bibtex url
\setcounter{biburllcpenalty}{7000}
\setcounter{biburlucpenalty}{8000}

%% The bibliography
\printbibliography[heading=bibintoc]

\label{theend} % the last page of the thesis

% List of acronyms
\printglossary[type=\acronymtype,title={\acrlistname}]

% Glossaries
\printglossary

%% Appendix
% !TEX root = ../thesis.tex

\chapter*{\appendixlistname}
\addcontentsline{toc}{chapter}{\appendixlistname}

\begin{description}
    \item[\appendixname{} A] Flowcharts
\end{description}

\appendix
\renewcommand\chaptername{\appendixname}
% !TEX root = ../thesis.tex

\chapter{Flowcharts}

\section{Short-term linear prediction}

\section{Long-term linear prediction}

\section{Extended long-term linear prediction}


% zivotopis autora
%\curriculumvitae\protect
%Táto časť\/ je nepovinná. Autor tu môže uviesť\/ svoje biografické
%údaje, údaje o~záujmoch, účasti na~projektoch, účasti na~súťažiach,
%získané ocenenia, zahraničné pobyty na~praxi, domácu prax, publikácie
%a~pod.

\end{document}
